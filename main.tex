% !TEX program = xelatex
\documentclass[a4paper,12pt]{article}

% Essential packages
\usepackage{geometry}
\usepackage{graphicx}
\usepackage{xeCJK}
\usepackage{titlesec}
\usepackage{booktabs}
\usepackage{listings}
\usepackage{float}
\usepackage{hyperref}
\usepackage{enumitem}
\usepackage{fancyhdr}
\usepackage{lastpage}

% Page geometry
\geometry{
    top=25mm,
    bottom=25mm,
    left=25mm,
    right=25mm
}

% Header and footer style
\pagestyle{fancy}
\fancyhf{}
\fancyhead[R]{\thepage\ of \pageref{LastPage}}
\renewcommand{\headrulewidth}{0pt}

% Title formatting
\titleformat{\section}
  {\normalfont\Large\bfseries}{\thesection}{1em}{}
\titleformat{\subsection}
  {\normalfont\large\bfseries}{\thesubsection}{1em}{}

% Code listing style
\lstset{
    basicstyle=\small\ttfamily,
    numbers=left,
    numberstyle=\tiny,
    frame=single,
    breaklines=true
}

% Document info
\title{
    \includegraphics[width=0.4\textwidth]{ysu-logo.png}\\[1cm] % logo
    {\LARGE Proposal For ASC 25}\\[0.5cm]
    \large Team Name: \texttt{[Team Name]}\\[1cm] % 使用占位符
    \begin{tabular}{c}
        \large Team Member 1 \\
        \large Team Member 2 \\
        \large Team Member 3 \\
        \large Team Member 4 \\
        \large Team Member 5
    \end{tabular}
}
\author{}
\date{\vfill January 8, 2025}

\begin{document}

\maketitle
\thispagestyle{empty}
\vfill
\begin{center}
    \textit{A proposal submitted to the ASC 25 committee}
\end{center}
\newpage

\tableofcontents
\addtocontents{toc}{\protect\enlargethispage{\baselineskip}}
\newpage

\section{Brief Background Description of Supercomputing Activities}

\subsection{Hardware and Software Platforms}
Our university established a high-performance computational (HPC) cluster in 2025 to address the growing demands in scientific research and industrial applications. Here's an example of how to include technical specifications:

\begin{table}[H]
\centering
\caption{Hardware Configuration of YSU HPC Cluster}
\begin{tabular}{llll}
\toprule
Item & Name & Configuration & Number \\
\midrule
Login Node & [Model] & CPU: [Specs] & 1 \\
Compute Node & [Model] & CPU: [Specs] & 10 \\
GPU Node & [Model] & GPU: [Specs] & 2 \\
\bottomrule
\end{tabular}
\end{table}

\subsection{Example Code Block}
Here's how to include code samples in the document:

\begin{lstlisting}[language=Python, caption=HPL Performance Testing Script]
def test_hpl_performance(problem_size, block_size):
    """
    Test HPL performance with given parameters
    """
    results = []
    for size in problem_size:
        perf = run_hpl_benchmark(size, block_size)
        results.append((size, perf))
    return results
\end{lstlisting}

\subsection{Example Figure}
Example of including a figure with caption:

\begin{figure}[H]
\centering
% 可以插入图像文件,也可以使用 LaTeX 绘图包生成图像
% \includegraphics[width=0.8\textwidth]{example-performance-graph.png}
This is a placeholder for the performance graph
\caption{HPL Performance Scaling Analysis}
\label{fig:performance}
\end{figure}

\newpage

\section{Design of HPC System}

\subsection{Performance Analysis}
Example of including mathematical equations:

\begin{equation}
R_{peak} = N_{cores} \times N_{flops/cycle} \times F_{clock}
\end{equation}

Where:
\begin{itemize}
    \item $N_{cores}$ is the total number of CPU cores
    \item $N_{flops/cycle}$ is the number of floating-point operations per cycle
    \item $F_{clock}$ is the clock frequency in Hz
\end{itemize}

\newpage

\section{Introduction to the University's Activities in Supercomputing}

\subsection{Supercomputing-related Hardware and Software Platforms}
% ...existing content...

\subsection{Supercomputing-related Courses, Trainings, and Interest Groups}
% 在这里添加关于课程、培训和兴趣小组的描述

\subsection{Supercomputing-related Research and Applications}
% 在这里添加关于研究和应用的描述

\subsection{Key Achievements in Supercomputing Research}
% 在这里简要描述不超过两项的关键成就

\newpage

\section{Team Introduction}

\subsection{Team Setup}
% 在这里简要描述团队设置

\subsection{Team Members}
% 在这里介绍每个成员并附上照片,包括团队合影

\subsection{Team Motto}
% 在这里添加团队的座右铭或口号

\newpage

\section{Technical Proposal Requirements}

\subsection{Design of HPC System}

\subsubsection{Theoretical Design of an HPC Cluster}
% 在这里添加关于HPC集群理论设计的内容

\subsubsection{Software and Hardware Configurations}
% 在这里添加关于软件和硬件配置的内容

\subsubsection{Interconnection, Power Consumption, Performance Evaluation, and Architecture Analysis}
% 在这里添加关于互连、功耗、性能评估和架构分析的内容

\subsection{HPL and HPCG Benchmarks}

\subsubsection{Software Environment}
% 在这里添加关于软件环境的描述

\subsubsection{Performance Optimization and Testing Methods}
% 在这里添加关于性能优化和测试方法的内容

\subsubsection{Performance Measurement and Problem/Solution Analysis}
% 在这里添加关于性能测量和问题/解决方案分析的内容

\subsubsection{In-depth Analysis of HPL and HPCG Algorithms and Source Codes}
% 在这里添加关于HPL和HPCG算法和源代码的深入分析

\subsection{Optimization for AlphaFold3 Inference}

\subsubsection{GPU Inference Optimization}
% 在这里添加关于优化GPU推理过程以最小化推理时间的内容

\subsubsection{CPU Inference Optimization}
% 在这里添加关于将推理代码库从GPU迁移到CPU架构并优化以最小化CPU推理时间的内容

\subsubsection{Inference Results}
% 在这里添加关于在GPU和CPU上分别运行所有十二个输入案例并上传结果的内容

\subsection{RNA m5C Modification Site Detection and Performance Optimization Challenge}

\subsubsection{Workflow Description}
% 在这里添加关于工作流描述文件的内容

\subsubsection{m5C Sites File}
% 在这里添加关于m5C位点文件的内容

\subsubsection{Software Packaging}
% 在这里添加关于将整个工作流打包成单个软件工具或容器的内容

\subsubsection{Performance Optimization}
% 在这里添加关于记录和提交从“cutseq”开始到工作流结束的时间的内容

\newpage

\section{Additional Materials}
% 在这里添加关于HPL输出文件、HPCG输出文件、AlphaFold3推理挑战所需文件和RNA m5C挑战所需文件的内容

\newpage

\appendix
\section{Additional Technical Details}
\subsection{Configuration Files}
Example of including configuration files:

\begin{lstlisting}[language=bash, caption=HPL Configuration File]
# Sample HPL.dat
HPL.out      output file name
6            device out (6=stdout,7=stderr,file)
1            # of problems sizes (N)
29000        Ns
1            # of NBs
256          NBs
0            PMAP process mapping (0=Row-,1=Column-major)
1            # of process grids (P x Q)
2            Ps
2            Qs
16.0         threshold
1            # of panel fact
2            PFACTs (0=left, 1=Crout, 2=Right)
\end{lstlisting}

\newpage

\section{References}
\begin{thebibliography}{9}
\bibitem{example}
Author, \textit{Title of the Book}, Publisher, Year.
\end{thebibliography}

\newpage

\end{document}